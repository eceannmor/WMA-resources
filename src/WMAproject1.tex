\documentclass[10pt]{extarticle}

\usepackage{fullpage}
\usepackage{enumitem}
\usepackage{subcaption}

% language and encoding
\usepackage{polski}
\usepackage[utf8]{inputenc}
\usepackage[english]{babel}

% metadata and link styling
\usepackage[
  pdfauthor={Ekre Ceannmor},
  pdftitle={WMA Project 1},
  pdfkeywords={},
  pdfproducer={Latex with hyperref},
  pdfcreator={tectonic},
  colorlinks=true,
  allcolors=MidnightBlue
]{hyperref}

% colours
\usepackage[svgnames, table]{xcolor}
\usepackage{amsfonts}
\usepackage{amsmath}
\usepackage{amssymb}

% footnotes in tables and figures
\usepackage{tablefootnote}
\usepackage{threeparttable}

% [H] specifier
\usepackage{float}

% specifying landscape mode for a page
\usepackage{pdflscape}

% external graphics
\usepackage{graphicx}
\graphicspath{ {./resources/} }

% table of contents styling
\usepackage[nottoc]{tocbibind}
\settocbibname{References}
\setcounter{tocdepth}{5}

\usepackage{bookmark}

% layout
\topmargin = -1.5cm
\textheight = 24cm

% block highlighting
\usepackage[framemethod=tikz]{mdframed}

% pretty tables
\usepackage{booktabs}

\pagenumbering{gobble}

% https://tex.stackexchange.com/questions/177202/booktabs-and-row-color
\colorlet{tableheadcolor}{gray!25} % Table header colour = 25% gray
\newcommand{\headcol}{\rowcolor{tableheadcolor}} %he
\colorlet{tablerowcolor}{gray!10} % Table row separator colour = 10% gray
\newcommand{\rowcol}{\rowcolor{tablerowcolor}} %
% Command \topline consists of a (slightly modified) \toprule followed by a \heavyrule rule of colour tableheadcolor (hence, 2 separate rules)
\newcommand{\topline}{\arrayrulecolor{black}\specialrule{0.1em}{\abovetopsep}{0.5pt}%
  \arrayrulecolor{tableheadcolor}\specialrule{\belowrulesep}{0pt}{-3pt}%
  \arrayrulecolor{black}
}
% Command \midline consists of 3 rules (top colour tableheadcolor, middle colour black, bottom colour white)
\newcommand{\midline}{\arrayrulecolor{tableheadcolor}\specialrule{\aboverulesep}{-1pt}{0pt}%
  \arrayrulecolor{black}\specialrule{\lightrulewidth}{0pt}{0pt}%
  \arrayrulecolor{white}\specialrule{\belowrulesep}{0pt}{-3pt}%
  \arrayrulecolor{black}
}
% Command \rowmidlinecw consists of 3 rules (top colour tablerowcolor, middle colour black, bottom colour white)
\newcommand{\rowmidlinecw}{\arrayrulecolor{tablerowcolor}\specialrule{\aboverulesep}{0pt}{0pt}%
  \arrayrulecolor{black}\specialrule{\lightrulewidth}{0pt}{0pt}%
  \arrayrulecolor{white}\specialrule{\belowrulesep}{0pt}{0pt}%
\arrayrulecolor{black}}
% Command \rowmidlinewc consists of 3 rules (top colour white, middle colour black, bottom colour tablerowcolor)
\newcommand{\rowmidlinewc}{\arrayrulecolor{white}\specialrule{\aboverulesep}{0pt}{0pt}%
  \arrayrulecolor{black}\specialrule{\lightrulewidth}{0pt}{0pt}%
  \arrayrulecolor{tablerowcolor}\specialrule{\belowrulesep}{0pt}{0pt}%
\arrayrulecolor{black}}
% Command \rowmidlinew consists of 1 white rule
\newcommand{\rowmidlinew}{\arrayrulecolor{white}\specialrule{\aboverulesep}{0pt}{0pt}%
\arrayrulecolor{black}}
% Command \rowmidlinec consists of 1 tablerowcolor rule
\newcommand{\rowmidlinec}{\arrayrulecolor{tablerowcolor}\specialrule{\aboverulesep}{0pt}{0pt}%
\arrayrulecolor{black}}
% Command \bottomline consists of 2 rules (top colour)
\newcommand{\bottomline}{\arrayrulecolor{white}\specialrule{\aboverulesep}{0pt}{-2pt}%
\arrayrulecolor{black}\specialrule{\heavyrulewidth}{0pt}{\belowbottomsep}}%
\newcommand{\bottomlinec}{\arrayrulecolor{tablerowcolor}\specialrule{\aboverulesep}{0pt}{0pt}%
\arrayrulecolor{black}\specialrule{\heavyrulewidth}{0pt}{\belowbottomsep}}%

\newcommand{\varQuad}[1]{\hskip#1em\relax}

% Code blocks
\usepackage{listings}
\definecolor{codegreen}{rgb}{0,0.6,0}
\definecolor{codegray}{rgb}{0.5,0.5,0.5}
\definecolor{codepurple}{rgb}{0.58,0,0.82}
\lstdefinestyle{default}{
  commentstyle=\color{codegreen},
  numberstyle=\tiny\color{codegray},
  stringstyle=\color{codepurple},
  basicstyle=\footnotesize\ttfamily,
  keywordstyle=\bfseries\color{magenta},
  backgroundcolor=\color{gray!10!white},
  breakatwhitespace=false,
  breaklines=true,
  keepspaces=true,
  numbers=left,
  numbersep=5pt,
  showspaces=false,
  showstringspaces=false,
  showtabs=false,
  tabsize=2,
  %=!=!=!=!=!=!=!=!=
  language=Python,
  %=!=!=!=!=!=!=!=!=
  frame=tb
}
\lstset{style=default}

% More tightly-packed text
\usepackage{setspace}

% Better indentation control
\usepackage{parskip}

\newcommand{\titletext}{Project 1}
\newcommand{\subtitletext}{Colour identification and tracking}

%===============================================================================
\begin{document}

{\Large \textbf{\titletext}} \\
{\large \subtitletext} \\
12 points total

%===============================================================================
\section*{Program 1}
\label{sec:program1}
Find the red ball in the picture \texttt{red\_ball.jpg}. \\
Mark the area identified as a "red ball" and its centre of gravity. \\
Draw text "Red ball" near the centre of gravity.

\begin{figure}[H]
  \centering
  \begin{subfigure}{.5\textwidth}
    \centering
    \includegraphics[width=.8\linewidth]{red_ball}
  \end{subfigure}%
  \begin{subfigure}{.5\textwidth}
    \centering
    \includegraphics[width=.8\linewidth]{red-ball-output}
  \end{subfigure}
  \caption{Example output of Program 1}
\end{figure}

\subsection*{Grading}
1 point each for the following:
\begin{itemize}[noitemsep]
  \item Correct import of required libraries.
  \item Correct import of the image file.
  \item Handling of edge cases related to file import.
  \item Conversion to HSV colour space.
  \item Tuning of colour masks to identify the ball.
  \item Use of binary operations to combine masks (at least 2 masks).
  \item De-noising using morphological operations.
  \item Correct centre of gravity and text placement.
\end{itemize}
8 points total.

%===============================================================================
\section*{Program 2}
\label{sec:program2}
Use the algorithm from Program 1 to identify the red ball in the video \texttt{rbg\_ball\_720.mp4}. \\
The position should update every frame, and the video should play continuously inside the same window. \\
If the ball is not in view of the camera, display the last position where it was seen (with some modification to the marker to indicate that the ball is not there).

\subsection*{Grading}
\begin{itemize}[noitemsep]
  \item 1 point - Correct import of the video file and handling of the edge cases.
  \item 2 points - Application of the algorithm from Program 1 to each frame of the video.
  \item 1 point - Handling of edge cases when the ball is not in view.
\end{itemize}
4 points total.

\end{document}